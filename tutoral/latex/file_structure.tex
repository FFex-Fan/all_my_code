% 导言区

% 可以使用 \documentclass[12pt]{article} 来设置全局字体大小为 12pt
\documentclass{article} % book, report, letter
%\documentclass{ctexart} % 可以通过定义 ctexart 来代替 使用 ctex 包


\usepackage{ctex}
\usepackage{graphicx} % 导入插图包
\usepackage{amsmath}
\usepackage{amssymb}
\graphicspath{{figure/}, {img/}} % 图片在当前目录下的 figures 和 img 文件夹中
%\usepackage{gensymb} % 提供了度数符号、弧度符号、特殊符号(欧姆、微米)等


% \newcommand{名称}{指令}
\newcommand{\degree}{^\circ} % 定义 \degree 含义
\newcommand{\myfont}{\itshape\bfseries\sffamily} %定义字体命令


\title{\heiti 勾股定理} % 中文前可以指定字体
\author{\kaishu 张三}

\date{\today}


% 正文区
\begin{document}
	\maketitle % 上述生成的标题、作者、日期需要加上该语句才会显示
	
	\section{结构介绍}
	hello \LaTeX \\ % 换行可以使用 \\ 或者空一行来实现
	
	% $$ 生成的是公式块,后面没有编号
	Let $f(x)$ be define by the formula $$f(x) = 3x^2 + x - 1$$ 	which is a polynomial of degree 2. \\
	
	%%%%%%%%%%%%%%%%%%%%%%%%%%%%%%%%%%%%%%%%%%%%%%%%%%%%%%%%%%%%%%%%%%%%%%%%%%%%%%%%%%%%%%%%%%%%%%%%%%%%
	
	\section{设置中文}
	
	% 使用中文需要使用 xlatex 进行编译,并且在导言区引入 ctex:
	%	方式一: \documentclass{ctexart}
	%	方式二: \documentclass{article} 
	%		   \usepackage{ctex}

	勾股定理可以用现代语言表述如下:
	
	直角三角形斜边的平方等于两腰的平方和。
	
	% \degree 为定义,因此会报错,故需要引入 gensymb 包或者定义该命令
	可以用符号语言表述为: 设直角三角形 $ABC$, 其中 $\angle C=90\degree$, 则有:
	
	\begin{equation} % equation 环境生成的公式后面会有编号
		AB^2 = BC^2 + AC^2
	\end{equation}
	
	%%%%%%%%%%%%%%%%%%%%%%%%%%%%%%%%%%%%%%%%%%%%%%%%%%%%%%%%%%%%%%%%%%%%%%%%%%%%%%%%%%%%%%%%%%%%%%%%%%%%
	
	\section{设置字体}
	% 可以使用 {} 来限定字体声明的范围


	% 字体族:罗马字体、无衬线字体、打字机字体
	% 罗马字体:笔画起始处有装饰
	% 无衬线字体:笔画起始处无装饰
	% 打字机字体:即,等宽字体
	
	% 字体族设置(罗马字体、无衬线字体、等宽字体)
	\textrm{Roman Family} \textsf{Sans Serif Family} \texttt{Typewriter Family}
	
	% 或:
	\rmfamily Roman Family {\sffamily Sans Serif Family} {\ttfamily Typewriter Family} 

	
	% 字体系列设置(粗细、宽度)
	\textmd{Medium Series} \textbf{Boldface Series}
	
	{\mdseries Medium Series} {\bfseries Boldface Series}
	
	% 字体形状(直立、斜体、伪斜体)
	\textup{Upright Shape} \quad \textit{Italic Shape} \quad \textsl{Slanted Shape} \quad \textsc{Small Caps Shape}
	
	{\upshape Upright Shape} \quad {\itshape Italic Shape} \quad {\slshape Slanted Shape} \quad {\scshape Sall Caps Shape}
	
	
	
	% 中文字体
	{\songti 宋体} \quad {\heiti 黑体} \quad 	{\fangsong 仿宋} \quad {\kaishu 楷书} 
	
	中文字体的\textbf{粗体}与\textit{斜体}

	% 字体大小
	{\tiny  Hello}\\
	{\scriptsize  Hello}\\
	{\footnotesize  Hello}\\
	{\small  Hello}\\
	{\normalsize  Hello}\\
	{\large  Hello}\\
	{\LARGE  Hello}\\
	{\huge  Hello}\\
	{\Huge  Hello}\\
	
	{\zihao{0} 你好!} % 设置中文字体大小
	
	
	
	{\myfont Suggest Use Self-Font}
	
	%%%%%%%%%%%%%%%%%%%%%%%%%%%%%%%%%%%%%%%%%%%%%%%%%%%%%%%%%%%%%%%%%%%%%%%%%%%%%%%%%%%%%%%%%%%%%%%%%%%%
	
	\section{文档结构}
	\subsection{引言}
	中国人口模式的转变发生于民国时期 关于民国的进步,我只讲两个过去人们比较忽略的问题。	一是人口模式。如前所述,传统时代人口的增减是王朝兴衰的显示器。
	
	中国人口模式的转变发生于民国时期 关于民国的进步,我只讲两个过去人们比较忽略的问题。	一是人口模式。如前所述,传统时代人口的增减是王朝兴衰的显示器。\par
	中国人口模式的转变发生于民国时期 关于民国的进步,我只讲两个过去人们比较忽略的问题。	一是人口模式。如前所述,传统时代人口的增减是王朝兴衰的显示器。\\中国人口模式的转变发生于民国时期 关于民国的进步,我只讲两个过去人们比较忽略的问题。	一是人口模式。如前所述,传统时代人口的增减是王朝兴衰的显示器。
	\subsection{实验方法}
	\subsection{实验结果}
	\subsubsection{数据}
	\subsubsection{图表}
	\subsubsection{实验过程}
	\subsection{结论}
	\subsection{致谢}	
	
	
	
	%%%%%%%%%%%%%%%%%%%%%%%%%%%%%%%%%%%%%%%%%%%%%%%%%%
	
	
	\section{特殊字符}
	
	\subsection{空白符号}
	实在学不了唱歌,能把主子伺候好,跟老母亲我一起去淘宝直播 卖宠物用品也是极好的啊,我学生家里就是卖宠物用品的,中日混血,家里有一群猫狗,一墙的仓鼠笼子(我看过照片,大概有几十个的样子)
	

	% 1em (当前字体中M的宽度)
	quad:a \quad b 
	
	% 2em
	qquad:a \qquad b 
	
	% 约为 1/6 个em
	thinspace:a\,b a \thinspace b 
	
	
	% 0.5em
	enspace:a \enspace b 
	
	% 空格
	空格 a\ b 
	
	% 硬空格
	硬空格 a~b
	
	% 1pc = 12pt = 4.218mm
	a \kern 1pc b
	
	a \kern -1em b
	 
	a \hskip 1em b 
	
	a \hspace{35pt} b 
	
	% 占位宽度
	占为宽度 a \hphantom{xyz} b
	
	
	% 弹性长度
	弹性长度 a \hfill b 


	
	
	\subsection{\LaTeX 控制符}
	\# \quad \$ \quad \{ \quad \} \quad \~{} \quad \_{} \quad \^{} \quad \textbackslash
	
	\subsection{排版符号}
	\S \quad \P \quad \dag \quad \ddag \quad \copyright \quad \pounds
	
	\subsection{\TeX 标志符号}
	\TeX{} \quad \LaTeX{} \quad \LaTeXe{}
	
	\subsection{引号}
	` \quad ' \quad `` \quad '' \quad  ``被引号包裹'' %  `表示单引号的左边,'表示单引号的右边
	
	\subsection{连字符}
	- \quad -- \quad --- % 分别生成短、中、长三种连字符
	
	\subsection{非英文字符}
	\oe \quad \OE
	
	\subsection{重音符号}
	\`o \quad \'o
	
	
	
	\section{插图}
	
	
	% 导言区: \usepackage{graphicx}
	% 		 \graphicspath{{figure/}, {img/}} % 设置文件查找路径
	% 语  法: \inncludegraphics[< 选项 >]{< 文件名 >} 
	% 格  式: EPS, PDF, PNG, JPEG, BMP
	
	\includegraphics{road.jpg}
	
	\includegraphics{sdu}
	
	
	
	\includegraphics[scale=0.1]{road.jpg}
	
	\includegraphics[scale=0.1]{sdu}
	
	!!!!!!!!!!!!!!!!!!!!!!!
	
	\includegraphics[height=2cm]{road.jpg}
	
	\includegraphics[height=2cm]{sdu}

	\includegraphics[width=2cm]{road.jpg}
	
	\includegraphics[width=2cm]{sdu}	
	
	
	!!!!!!!!!!!!!!!!!!!!!!!
	
	\includegraphics[height=0.1\textheight]{road.jpg}
	
	\includegraphics[height=0.1\textheight]{sdu}

	\includegraphics[width=0.2\textwidth]{road.jpg}
	
	\includegraphics[width=0.2\textwidth]{sdu}	
	
	
	!!!!!!!!!!!!!!!!!!!!!!!
	
	
	
	\includegraphics[angle=-45, width=0.2\textwidth]{road.jpg}
	
	\includegraphics[angle=45, width=0.2\textwidth]{sdu}	
	
	
	
	
	
	\section{表格}
	
%		\begin{tabular}[<垂直对齐方式>]{<列格式说明>}
%			<表项> & <表项> & <表项> \\
%		\end{tabular}
%		l - 本列左对齐
%		c - 本列居中对齐
%		r - 本列右对齐
% 		p{<宽>} - 本列宽度固定,能够自动换行
	
	\begin{tabular}{|l|c|r|c|} % 在后面指定列数
		\hline % 每行之间使用横线进行分割
		姓名 & 语文 & 数学 & 外语 \\ % 使用 \\ 进行换行(添加行)
		\hline
		张三 & 78 & 134 & 123 \\
		\hline
	\end{tabular}


	\section{浮动体}
%		可以对图表进行浮动
%	图:
		见图\ref{img_sdu}
		\begin{figure}[htbp]
			\centering % 使得以下内容居中
			\includegraphics[scale=0.4]{sdu}
			\caption{这是sdu} \label{img_sdu}
		\end{figure}
		
		
		\newpage
		
		在\LaTeX{}中也可以使用表\ref{tab-score}所示的表格
		\begin{table}[htbp]
			\centering
			\caption{成绩单}\label{tab-score}
				\begin{tabular}{|l| c| c| c|  r|}%会有5列,指定每列的居中形式,|表示每列中间有竖线分开
				\hline%每行之间由横线分开
				姓名&语文&数学&外语&政治\\%\\表示换行
				\hline
				张三&87&120&25&36\\
				\hline
				张1&87&120&25&36\\
				\hline
				张2&87&120&25&36\\
				\hline
			\end{tabular}
		\end{table}
	
	\newpage
		
	\section{数学公式}
		\subsection{行内公式}
			\subsubsection{美元符号}
				交换律$a+b=b+a$,如$1+2=2+1$
			\subsubsection{小括号}
				交换律\(a+b=b+a\),如\(1+2=2+1\)
			\subsubsection{math环境}
				交换律
					\begin{math}
						a+b=b+a
					\end{math}
		\subsection{上下标}
			\subsubsection{上标}
				$2x^2+3x+5=6$
			\subsubsection{下标}
				$a_0,a_1,a_{100}$
		\subsection{希腊字母}
			$\alpha$
			$\beta$
			$\gamma$
			$\epsilon$
			$\pi$
			$\omega$
	
			$\Gamma$
			$\Delta$
			$\Theta$
			$\Pi$
			$\Omega$
		\subsection{数学函数}
			$\log$
			$\sin$
			$\cos$
			$\arccos$
			$\arcsin$
			$\ln$
	
			$\sin^2x+\cos^2x=1$
	
			$\sqrt{2}$
			$\sqrt{x^2+y^2}$
			$\sqrt{2+\sqrt{2}}$
			$\sqrt[4]{x}$
	\subsection{分式}
		大约是原体积的$3/4$
		大约是原体积的$\frac{3}{4}$
	\subsection{行间公式}
		\subsubsection{使用 \$\$}
	        $$2x^2+5x+3=6$$	
		\subsubsection{使用 displayment}
			\begin{displaymath}
				2x^2+5x+3=6	
			\end{displaymath}
		\subsubsection{使用 \textbackslash[ }
			\[2x^2+5x+3=6\]
		\subsubsection{自动编号公式}
			交换律见式\ref{eq:commutative}
			\begin{equation}
				a+b=b+a \label{eq:commutative}
			\end{equation}
    	\subsubsection{不带自动编号公式}
    		\begin{equation*}
    			a+b=b+a
    		\end{equation*} % 需要使用\usepackage{amsmath}
    
    
    
    
    
    
    
    \newpage
    \section{矩阵公式}
   		\[
   			\begin{matrix}%&分列 \\分行
  				0&1\\
  				1&0
			\end{matrix}\quad
		\]

		\[
			\begin{pmatrix}%括号包裹的矩阵
 				0&1\\
				1&0
			\end{pmatrix}
		\]	

		\[
			\begin{vmatrix}%长竖线包裹的矩阵
				0&1\\
				1&0
			\end{vmatrix}
		\]

		\[
			\begin{bmatrix}%长中括号包裹的矩阵
				0&1\\
				1&0
			\end{bmatrix}
		\]


		\[
			\begin{pmatrix}%括号包裹的矩阵
				a_{11}^2&a_{12}^2&a_{13}^2\\
				0&a_{22}&a_{33}
			\end{pmatrix}
		\]

		\[
			\begin{bmatrix}%长中括号包裹的矩阵
				a_{11}&\dots & a_{1n}\\
				&\ddots & \vdots\\
				&       & a_{nn}
			\end{bmatrix}_{n \times n} % 后面加上下角标
		\]

		\[
			\begin{pmatrix}%分块矩阵(矩阵嵌套)
				\begin{matrix}
					1&0\\0&1
				\end{matrix}
				& \text{\Large 0}\\
				
				\text{\Large 0}&
				\begin{matrix}
					1&0\\0&1
				\end{matrix}
			\end{pmatrix}
		\]

		\[
			\begin{pmatrix}%括号包裹的矩阵
				a_{11}&a_{12}&\cdots&a_{ln}\\
				&a_{22}&\cdots&a_{2n}\\
				&		&\dots &\vdots \\
				\multicolumn{2}{c}{\raisebox{1.3ex}[0pt]{\Huge 0}}
				&		&a_{nn}
			\end{pmatrix}
		\]


		\[
			\begin{pmatrix}%跨列的省略号:\hdotsfor{<列数>}
				1&\frac 12 &\dots &\frac ln \\
				\hdotsfor{4}\\
				m&\frac m2& \dots &\frac mn
			\end{pmatrix}
		\]

		%行内小矩阵(smallmatrix)环境
		复数$z=(x,y)$也可以用矩阵
		\begin{math}
	   		\left(%需手动加上左括号
   				\begin{smallmatrix}
					x& -y\\y&x
   				\end{smallmatrix}
			\right)%需手动加上右括号
		\end{math}来表示

		%array环境(类似表格环境tabular)
		\[
			\begin{array}{r|r}
				\frac 12&0\\
				\hline
				0& -\frac abc\\
			\end{array}
		\]
		
		
		
		% 在数学公式中处理中文时,需要使用\text命令
	
	
	
	
	
	
	\newpage
	\section{多行公式}
	%gather和gather*环境(可以使用\\换行)
	%带编号
	\begin{gather}
		a+b=b+a\\
		ab  ba
	\end{gather}
	
	%不带编号
	\begin{gather*}
	3+5=5+3\\
	3 \times 5=5\times 3
	\end{gather*}
	
	%在\\前使用\notetag阻止编号
	\begin{gather}
		3^2+4^2=5^2
	\end{gather}
	
	%align和align*环境(用&对齐)
	%带编号
	\begin{align}
		x &=t+\cos t+1\\
		y &=2 \sin t
	\end{align}
	%不带编号
	\begin{align*}
	x &=t+\cos t+1\\
	y &=2 \sin t
	\end{align*}
	
	%split环境(对齐采用align环境的方式,编号在中间)
	\begin{equation}
	\begin{split}
	\cos 2x &=\cos^2 x- \sin^2 x\\
	&=2\cos^2 x-1
	\end{split}
	\end{equation}		
	
	%case环境
	%每行公式中使用&分隔为两部分
	%通常表示值和后面的条件
	\begin{equation}
	D(x)=\begin{cases}
	1,& if  x \in \mathbb{Q};\\
	0,& if  x \in \mathbb{R} \setminus \mathbb{Q}
	\end{cases}
	\end{equation}
	
	
	
	
	
	
	
	
	
	
	
	
	
	
	
	
	
	
	
	
	
	
	
	
	
	
	
	
	
	
	
	
	
\end{document}